\chapter{Notas Personales}\label{chap:ab}


% #################################################
% #################################################
\section{Prefacio}

Al momento de escribir este documento he querido compartir en el mejor de los sentidos un problema que me he 
encontrado a lo largo de mi carrera profesional y mi visión personal para abordarla. Este documento 
no pretende fijar un dogma (religión o cualquier cosa similar), sino una idea que permee a través de las 
personas a diferentes situaciones y que sea capaz de adaptarse, es decir, lo más importante para quienes 
lean este documento es entender el concepto y que cada uno desarrolle de manera interna el concepto haciendo 
uso de la introspección, con esto mantengo vivo el espíritu de esta obra.

En primer lugar debo de reconocer a las fuentes de inspiración, indudablemente fue el Zen de Python, 
esta lectura marcó mi manera de trabajar y la forma de ver mi vida personal. En segundo lugar Scrum, 
la forma de abordar los flujos y la forma de responder al trabajo se me ha hecho francamente una genialidad. 
En una parte más personal es el Manifesto Cypherpunk y la forma de expresar la libertad, así como la filosofía 
oriental especialmente el Budismo Zen. Pero sin mayor duda la principal razón que me trajo al mundo de 
IT ha sido Linux y su filosofía alrededor, al escribir este documento es una deuda que tenía con toda la 
comunidad por compartir su conocimiento con todo el mundo sin esperar nada a cambio. En aras de la gratitud 
este documento está dedicado a todas esas personas, pero especialmente va dedicado a todas las personas que 
no han recibido reconocimiento ni gratitud necesaria por compartir su conocimiento y han quedado en el olvido.

\footnote{Nota: Las referencias a frameworks existentes (Scrum, ITIL, Lean, DevOps, etc.) se usan únicamente con fines comparativos y de integración conceptual. No constituyen reproducción ni adaptación de sus contenidos originales.}
\\ 
\\
\hrule
% #################################################
% #################################################
\section{Origen del framework}

Kanso nace de más de 10 años trabajando en IT desde on-premise hasta Cloud y SaaS en contextos de startups, 
enterprise, consultorías, freelance, donde observé el mismo patrón:

Cada nueva herramienta/proceso prometía eficiencia, pero el OPEX real crecía más rápido que el valor entregado. 
Las horas se consumían en integraciones frágiles, procesos redundantes y coordinación manual.

Existía un costo invisible que ningún TCO capturaba: el tiempo malgastado en fricciones evitables. 
Este framework sistematiza las técnicas que usé para:
\begin{itemize}
    \item Cuantificar ese costo (IK)
    \item Clasificar fricciones (inevitable vs evitable)
    \item Reducirlas metódicamente (Friction RCA → Documentar implementación)
\end{itemize}

Esta es la v1.0 pública. Se basa en experiencia real, pero necesita validación en contextos diversos. 
Si lo implementas, tu feedback ayudará a refinar métricas, roles y tiempos óptimos. Contexto:
\begin{itemize}
    \item Email: \href{mailto:contact@kansoframework.org}{contact@kansoframework.org}
    \item Website: \href{kansoframework.org}{kansoframework.org}
\\
\\
\end{itemize}

\hrule
% #################################################
% #################################################
\section{Ordo Fluens}

\begin{enumerate}
    \item \textit{Simplicidad no como ausencia de esfuerzo, sino como esfuerzo intencional.}
    \item \textit{El caos es complejidad descontrolada.}
    \item \textit{La complejidad sin conocimiento se convierte en caos.}
    \item \textit{La complejidad crea fricciones, y las fricciones son problemas.}
    \item \textit{La complejidad clara no es buena, pero la complejidad oculta es peligrosa.}
    \item \textit{Una buena solución reduce la complejidad.}
    \item \textit{Una mala solución aumenta la complejidad.}
    \item \textit{Lo inevitable es aceptado y ordenado.}
    \item \textit{Lo evitable se reduce o se encapsula en caminos claros.}
    \item \textit{Menos complejidad, menos caos.}
    \item \textit{Menos caos, más armonía.}
    \item \textit{Más armonía, más simplicidad.}
    \item \textit{Balance es mejor que equilibrio.}
    \item \textit{Armonía es mejor que balance.}
    \item \textit{Complejidad es inevitable, Simplicidad es intencional.}
\\
\\
\end{enumerate}

% #################################################
% #################################################
\section{Sobre la Naturaleza Abierta de Esta Obra}

Este framework se publica bajo licencia CC BY-SA 4.0 porque creo que el conocimiento operativo debe ser accesible para todos. He visto demasiadas ``metodologías propietarias'' que son simplemente sentido común envuelto en jerga y certificaciones costosas.

Kanso será exitoso no por control legal, sino por utilidad práctica. Si encuentras valor en estos conceptos, úsalos. Si los mejoras, compártelos. Si no te sirven, descártalos sin culpa.

No puedo controlar cómo interpretarás o aplicarás estas ideas. Solo puedo confiar en que quienes resuenen con el espíritu de simplicidad operativa lo mantendrán vivo en sus implementaciones.

Las ideas pertenecen al ecosistema de conocimiento colectivo. Este documento es solo un intento de organizarlas de manera útil.
\footnote{%
\textbf{Sobre Prior Art:} Kanso construye sobre ideas de Lean, Theory of Constraints, 
Agile, DevOps y Systems Thinking. Mi contribución específica es la sistematización de 
``fricción operativa'' como métrica y la metodología para reducirla (IK, Friction Ledger, 
Friction RCA). Las referencias a frameworks existentes se usan bajo fair use educativo, 
con respeto a sus creadores originales.
}


\begin{flushright}
\textit{— Eduardo López, octubre 2025}
\end{flushright}