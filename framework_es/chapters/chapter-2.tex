\chapter{El sistema}\label{chap:two}

% #################################################
% #################################################
\section{Fricción}

La fricción es el concepto principal de este marco de trabajo, y la podemos definir como la resistencia natural de un ecosistema a la introducción de un nuevo elemento al mismo. Para lo cual tenemos 2 tipos evitables e inevitables.

\textbf{Fricciones inevitables}\\
Son las que sí o sí tendrás al operar una solución, incluso con el mejor diseño.

Criterios:
\begin{enumerate}
    \item Naturaleza esencial: sin esa fricción, la solución no podría funcionar en tu contexto. Ej: aplicar parches de seguridad críticos.
    \item Obligación externa: impuesta por leyes, compliance o regulaciones. Ej: auditorías PCI, GDPR, ISO.
    \item Limitaciones técnicas universales: derivadas de la tecnología en sí, no de tu implementación. Ej: latencia física en redes de larga distancia.
    \item Esfuerzo proporcional: el costo de reducirla sería mayor al beneficio. Ej: entrenar pilotos para operar un avión → inevitable.
\end{enumerate}

\textit{Ejemplo TI:} Mantener certificados TLS actualizados. No hay forma de evitarlo, solo automatizarlo un poco.

\textbf{Fricciones evitables}\\
Son las que existen solo porque el diseño, proceso o integración no es óptimo.

Criterios:
\begin{enumerate}
    \item Redundancia innecesaria: dos procesos/herramientas hacen lo mismo. Ej: usar tres sistemas de monitoreo en paralelo.
    \item Falta de estandarización: cada equipo usa un lenguaje/tool distinto para lo mismo. Ej: Terraform + Ansible + scripts ad hoc.
    \item Manualidad repetitiva: tareas mecánicas que pueden automatizarse. Ej: copiar logs de un servidor a otro a mano.
    \item Errores recurrentes: mismas fallas una y otra vez → síntoma de proceso mal diseñado. Ej: pipelines que fallan por validaciones manuales.
    \item Burocracia excesiva: aprobaciones o pasos que no agregan valor real. Ej: 3 firmas para liberar un cambio trivial.
    \item Mala integración: adaptadores frágiles, procesos manuales para pasar datos entre sistemas.
\end{enumerate}

\textit{Ejemplo IT:} Un DevOps pasa 4h/semana ajustando scripts porque el sistema de tickets no se integra bien con el CI/CD. → evitable.

\begin{itemize}
    \item Inevitable: costo mínimo natural, ligado a operar cualquier solución.
    \item Evitable: sobrecosto (overhead) generado por mal diseño, redundancia o falta de automatización.
\end{itemize}

% #################################################
% #################################################
\subsection*{Identificar fricciones}

\begin{quote}
\textit{Hacer visible lo invisible. Detectar dónde se va el tiempo y por qué.}
\end{quote}

Señales típicas:
\begin{enumerate}
    \item Tareas manuales
    \item Handoffs para una tarea
    \item Integraciones frágiles
    \item Problemas y tareas recurrentes
    \item Flujos de trabajo lentos
    \item MTTD/MTTR altos
    \item Burocracia
    \item Seguridad
\end{enumerate}

Fuentes de evidencia:
\begin{enumerate}
    \item Tickets: tiempos de espera, re-trabajos, causa raíz.
    \item Repos/Pipelines: duraciones, aprobaciones, reintentos, fallos.
    \item Observabilidad: frecuencia de incidentes, downtime.
    \item Entrevistas rápidas (15 min): ``¿qué pasos te sobran?'', ``¿qué automatizarías primero?'' ``¿Qué necesitas para hacer tu trabajo más eficiente?''
\\
\\
\end{enumerate}


\hrule
% #################################################
% #################################################
\section{Costo de Complejidad}

El costo de complejidad es el costo total de fricciones de operar y/o introducir soluciones no plug-and-play 
\footnote{La soluciones plug and play son casi inexistentes pero se utiliza como termino de referecnia.} 
en un ecosistema. En otras palabras, el costo de complejidad es la suma de las fricciones inevitables (OPEX natural) 
y las fricciones evitables (overhead del OPEX) que surgen al operar soluciones no plug-and-play.
\\
\\
Desde otro punto de vista, el costo de complejidad es el costo total de gastos visibles y 
ocultos de fricciones que genera una solución al operar en un contexto y se define por las preguntas:

\begin{itemize}
    \item ¿Cuál es el costo real de implementar esta solución en mi ecosistema?
    \item ¿Cuál es el overhead de mi opex?
    \item ¿Qué tanto problema ha sido integrar esta nueva solución?
\end{itemize}

Ejemplo: Tengo mi negocio de mensajería local por lo cual solo opero vehículos, entonces deseo expandirme para 
lo cual quiero introducir un avión para poder realizar entregas interestatales, entonces ¿cuál sería el 
costo de complejidad añadir esta solución? La fricción de meter un avión en un ecosistema que estaba diseñado 
solo para vehículos terrestres, este el verdadero costo de complejidad resumido. 
En resumen, es el costo que genera la complejidad de las interacciones entre los elementos de un sistema
\\
\\
{\textbf{TCO vs Costo de Complejidad}}
\\
El TCO (Total Cost of Ownership) es el costo total de poseer y operar una solución durante todo su ciclo de vida. 
El Costo de Complejidad son 2 subconjuntos del TCO, estos son el OPEX (visible) y el overhead operativo causado por 
fricciones evitables (Overhead del OPEX usualmente oculto).
\\
\\
Ejemplo:
\begin{itemize}
\item Comprar una herramienta de monitoreo: TCO = licencias + servidores + soporte + costo de complejidad.
\item Costo de complejidad: ingenieros que operen la solución más tener que escribir adapters manuales 
para que se integren con tu sistema de alertas, más horas de troubleshooting porque rompe aplicaciones.
\\
\end{itemize} 


{\textbf{OPEX vs Costo de Complejidad}}
\\
OPEX es el gasto recurrente y operativo que permite que las soluciones funcionen día a día. 
El Costo de Complejidad es el OPEX más los gastos ocultos de operar una solución, es decir:
Cuando las fricciones evitables = 0, el costo de complejidad se reduce al OPEX natural (fricciones inevitables). 
El problema aparece cuando el esfuerzo operativo deja de ser natural y se convierte en un overhead que no genera 
valor directo: como integraciones deficientes, procesos redundantes o falta de automatización.
Otra manera de verlo es: 
\begin{center}
Costo de complejidad = OPEX natural + Overhead del OPEX (fricciones evitables)
\end{center}

Ejemplo:
\begin{itemize}
\item OPEX natural: Tienes un cluster de Kubernetes. Un DevOps cobra 100\% de su sueldo para operarlo.
\item OPEX natural + Overhead del OPEX: Mismo clúster, pero cada semana se rompe la integración con el 
monitoring y requiere horas extra de troubleshooting. Además, cada despliegue necesita coordinación manual entre 3 equipos.
\end{itemize} 

En otras palabras:

\begin{itemize}
    \item El salario ya estaba presupuestado; lo que cambia es que se malgasta en fricciones.
    \item No es un costo nuevo, es un costo de oportunidad perdido.
    \item Son horas de OPEX que no producen el valor esperado.
    \item Es el tiempo desperdiciado en fricciones innecesarias que no generan valor.
\end{itemize}

{\textbf{Diferencia Conceptual}}

\begin{table}[h]
\centering
\begin{tabular}{|p{3.5cm}|p{4.5cm}|p{6cm}|}
\hline
\textbf{Concepto} & \textbf{Naturaleza} & \textbf{Relación con el Costo de Complejidad} \\ \hline
TCO & Costo total de poseer una solución (CAPEX + OPEX + costos indirectos). & El Costo de Complejidad está dentro del TCO. Es una parte variable y dinámica que depende de las fricciones que genera la solución. \\ \hline
OPEX & Parte recurrente del TCO (operación diaria). & El Costo de Complejidad se manifiesta principalmente dentro del OPEX como sobrecarga o desperdicio operativo. \\ \hline
Costo de Complejidad & Efecto emergente dentro del TCO y del OPEX causado por fricciones evitables. & Puede entenderse como OPEX natural + overhead, o como la diferencia entre OPEX ideal y OPEX real. \\ \hline
\end{tabular}
\end{table}

\newpage
\hrule

% #################################################
% #################################################

\section{Anclaje cuantitativo de la fricción}

{\textbf{Definición general}}\\
Fricción operativa = tiempo improductivo invertido en resolver o compensar ineficiencias del sistema.\\
Costo de fricción = (horas perdidas / mes) $\times$ (costo hora promedio del rol afectado).
\\
\\
{\textbf{Escala de referencia (aplicable a cualquier dominio)}}
\\
\begin{table}[h]
\centering
\small
\begin{tabular}{|p{3cm}|p{2.5cm}|p{2cm}|p{2.5cm}|p{2cm}|p{2cm}|}
\hline
\textbf{Tipo de organización} & \textbf{Rol promedio} & \textbf{Costo hora estimado} & \textbf{Fricción promedio (h/mes por persona)} & \textbf{Costo mensual oculto} & \textbf{\% típico del OPEX desperdiciado} \\ \hline
Empresa SaaS / IT & Ingeniero / DevOps & \$50 USD & 20--40 h & \$1,000--\$2,000 & 10--25\% \\ \hline
Servicios financieros & Analista / Gestor & \$60 USD & 15--30 h & \$900--\$1,800 & 8--20\% \\ \hline
Manufactura & Supervisor / Técnico & \$35 USD & 10--25 h & \$350--\$875 & 5--15\% \\ \hline
Gobierno / Educación & Administrador & \$25 USD & 15--30 h & \$375--\$750 & 8--18\% \\ \hline
Retail / Operaciones & Coordinador / Logística & \$20 USD & 10--20 h & \$200--\$400 & 5--10\% \\ \hline
\end{tabular}
\end{table}

Estas cifras no buscan precisión contable, sino dar magnitud: incluso una fricción leve 
(20 horas improductivas/mes por empleado) representa entre 10 y 20\% del costo operativo 
total en la mayoría de los sectores.
\\
\\
Ejemplo:

Una empresa de servicios con 50 empleados administrativos y técnicos detecta que cada uno pierde, en promedio, 
15 horas al mes en tareas redundantes o esperas interdepartamentales.
Con un costo hora promedio de \$40 USD:

\begin{center}
$50 \times 15h \times 40 = 30{,}000$ USD/mes
\end{center}

Eso equivale a \$360,000 USD al año de OPEX improductivo.

Reducir esa fricción en 40\% con prácticas de Kanso (automatización, estandarización y fluidez) 
libera \$144,000 USD/año, sin agregar recursos ni cambiar de tecnología.
\\
\\
\textbf{¿Entonces?}

\begin{enumerate}
    \item ¿Por qué ahora?, responde a una verdad estructural: la eficiencia moderna ya no depende de agregar tecnología, sino de reducir el costo invisible de mantenerla funcionando junta.
    \item El anclaje cuantitativo da al framework una base empírica: una fricción no es un concepto filosófico, sino una pérdida económica recurrente y mensurable en cualquier entorno humano o técnico.
\\
\\
\end{enumerate}


\hrule
% #################################################
% #################################################
\section{Kanso Framework}

Kanso es una palabra japonesa que significa simplicidad o ausencia de adornos. 
Forma parte de los siete principios de la estética zen y representa la idea de que la verdadera belleza, 
claridad y eficiencia emergen cuando se elimina lo innecesario y solo queda lo esencial.
En el contexto del Kanso Framework, esta palabra encarna la filosofía central: reducir la 
complejidad a su mínima expresión, conservando únicamente lo que genera valor real.
El framework propone armonía operativa: menos fricción, menos desperdicio de tiempo y un OPEX más saludable.
El Kanso Framework tiene como objetivo optimizar el costo de complejidad en organizaciones modernas.

\begin{itemize}
    \item Identificar las fricciones inevitables y las evitables en la operación de soluciones.
    \item Reducir al máximo las fricciones evitables, como redundancias, procesos manuales innecesarios o integraciones defectuosas.
    \item Encapsular y gestionar la complejidad inevitable con métodos claros y predecibles.
    \item Mantener un OPEX saludable, asegurando que los equipos dediquen su tiempo y energía a crear valor y no a reparar fricciones.
\end{itemize}

En esencia, Kanso Framework ayuda a las organizaciones a transformar ecosistemas complejos en sistemas más simples, 
claros y sostenibles. El alma de este framework es mantener la simplicidad en entornos complejos. 
En una sola frase es el scrum para la eficiencia del OPEX.
\\
\\
\textbf{El Ethos Kanso}
\\
La implementación efectiva de Kanso requiere tres prácticas fundamentales:
\begin{enumerate}
    \item Diseño deliberado: Cada proceso debe tener un propósito medible, no existir por inercia.
    \item Visibilidad operativa: Documentar el flujo de trabajo real, no el teórico.
    \item Reducción de handoffs: Minimizar puntos de transferencia entre personas, equipos y sistemas.
\end{enumerate}
Adoptar Kanso es adoptar una cultura de simplicidad consciente.

{\epigraph{\textit{La complejidad técnica es inevitable. La simplicidad operativa requiere diseño deliberado.}}{}}

\hrule
% #################################################
% #################################################
\section{Índice Kanso}
\begin{quote}
\textit{La métrica del costo de complejidad.}
\end{quote}


El Índice Kanso (IK) es la métrica central del framework. Cuantifica qué porcentaje del gasto operativo (OPEX) se pierde en trabajo improductivo causado por fricciones evitables.

Se expresa como una proporción:

\[
IK = \frac{Overhead}{OPEX} \times 100
\]

Donde:
\begin{itemize}
    \item Overhead = horas desperdiciadas (fricciones evitables) × costo hora.
\end{itemize}

Interpretación IK
\begin{itemize}
    \item IK bajo (0--15\%) → Sistema eficiente (zona verde). La mayor parte del esfuerzo operativo se dedica a crear valor. Las fricciones inevitables están bajo control.
    \item IK moderado (15--25\%) → Ineficiencias moderadas (zona amarilla). El sistema funciona, pero hay fugas de eficiencia (duplicidades, manualidad, malas integraciones). Es momento de aplicar Kanso para reducir fricción.
    \item IK alto (>25\%) → Crisis operativa (zona roja). Una cuarta parte o más del gasto se pierde en overhead. Esto significa que la organización trabaja más para mantener el sistema que para hacer avanzar el negocio.
\end{itemize}

Comparación con métricas existentes:


Ejemplo: Un sistema con TCO bajo puede tener IK alto si su operación requiere mucho trabajo manual."

\begin{itemize}
    \item TCO mide costo total de propiedad (CAPEX + OPEX)
    \item IK mide qué porción del OPEX es desperdicio oculto
    \item Ejemplo: Un sistema con TCO bajo puede tener IK alto si su operación requiere mucho trabajo manual."    
\\
\\
\end{itemize}

\hrule
% #################################################
% #################################################
\section{Friction Ledger}
\begin{quote}
\textit{El inventario vivo de fricciones.}
\end{quote}

El Friction Ledger (registro de fricciones) funciona como un backlog especializado donde se documentan y cuantifican ineficiencias operativas. 
A diferencia de un backlog tradicional enfocado en features, el Ledger registra overhead operativo con su costo asociado.

\textbf{Objetivo}
\begin{itemize}
    \item Cuantificar: convertir horas desperdiciadas en costo monetario.
    \item Clasificar: separar lo inevitable (OPEX natural) de lo evitable (overhead).
    \item Priorizar: identificar qué fricciones deben eliminarse, automatizarse o rediseñarse primero.
    \item Comunicar: ofrecer a negocio y ala tecnología un lenguaje común para entender la complejidad.
\end{itemize}

Estructura del ledger. Cada entrada debe contener al menos:

\begin{table}[h]
\centering
\scriptsize
\begin{tabular}{|l|p{2.2cm}|l|l|l|l|p{1.8cm}|p{2.2cm}|}
\hline
ID & Fricción & Clasif. & Tiempo (h/mes) & Costo/h & Costo/mes & Evidencia & Acción \\ \hline
F-001 & Integración CI/CD rompe credenciales cada semana & Evitable & 40 & \$50 & \$2,000 & Logs, tickets Jira & Automatizar rotación con Vault \\ \hline
F-002 & Auditoría PCI trimestral & Inevitable & 20 & \$70 & \$1,400 & Req. legal & Optimizar docs \\ \hline
F-003 & Aprobar cambios menores requiere 3 firmas & Evitable & 15 & \$60 & \$900 & Change mgmt & Policy as code \\ \hline
\end{tabular}
\end{table}

\textbf{Modo de uso}

Detección de Fricciones
\begin{enumerate}
    \item Revisar tickets, logs, pipelines y entrevistas rápidas.
    \item Anotar cada fricción detectada.
\end{enumerate}

Cálculo
\begin{enumerate}
    \item Tiempo perdido × costo hora = costo mensual/anual de esa fricción.
\end{enumerate}

Clasificación
\begin{enumerate}
    \item Inevitable → OPEX natural.
    \item Evitable → overhead de complejidad.
\end{enumerate}

Acción
\begin{enumerate}
    \item Documentar la intervención propuesta (eliminar, estandarizar, automatizar, rediseñar).
\end{enumerate}

Actualización continua
\begin{enumerate}
    \item El ledger debe mantenerse vivo, como un libro contable: cada mes se agregan, cierran o ajustan fricciones.
\end{enumerate}

\textbf{Valor ejecutivo}
\begin{itemize}
    \item Para el CFO: El Friction Ledger traduce la complejidad en dinero perdido.
    \item Para el CTO: Expone cuántas horas de ingenieros se malgastan en fricciones evitables.
    \item Para los equipos: Visibiliza la carga de trabajo improductivo y da argumentos para priorizar automatización o estandarización.
\end{itemize}

{\epigraph{\textit{El Friction Ledger cuantifica el costo operativo oculto al convertir horas perdidas en impacto económico medible.}}{}}


\hrule
% #################################################
% #################################################
\section{Friction RCA}
\begin{quote}
\textit{Entender la causa es más simple que combatir el síntoma.}
\end{quote}

\textbf{Objetivo} 
\begin{itemize}
    \item El Friction RCA(Root Cause Analysis) tiene como objetivo descubrir por qué existe una fricción.
    \item Mientras el Friction Ledger responde a ``qué ocurre'' y ``cuánto cuesta'', el Friction RCA responde a ``por qué ocurre'' y ``qué debe cambiar en el flujo''. En general es mejor reordenar que rediseñar.
\end{itemize}


\textbf{Modo de uso}

Deteccion 
\begin{enumerate}
    \item ¿En qué punto del proceso ocurre la fricción?
    \item ¿Qué lo precede y qué lo sigue? ¿Qué equipos, sistemas o personas intervienen?
    \item (Mapa mínimo de flujo: entrada → transformación → salida)
\end{enumerate}

Análisis causal (las tres raíces). Aplicar el principio de los ``3 Whys'' (por qué ×3).

\begin{center}
\begin{tabular}{l|l|p{5cm}|p{4cm}|p{3cm}|}
\hline
ID & Causa raíz & Descripción & Evidencia & Nivel \\ \hline
F-001 & 1 & La rotación de credenciales es manual & Logs de fallos & Técnica \\ \hline
F-001 & 2 & No existe ownership claro del secret store & Entrevista con DevOps & Organizacional \\ \hline
F-001 & 3 & La política de rotación no está automatizada ni auditada & Documento de seguridad & Proceso \\ \hline
\end{tabular}
\end{center}

Contramedidas propuestas. Cada causa raíz debe tener una respuesta proporcional y Kanso-alineada:

\begin{center}
\begin{tabular}{|p{3.5cm}|p{5cm}|p{3.5cm}|}
\hline
Tipo de causa & Contramedida Kanso & Regla aplicable \\ \hline
Técnica & Automatizar rotación con Vault & Automatización \\ \hline
Organizacional & Definir responsable de secretos & Estandarización \\ \hline
Proceso & Incorporar rotación en pipeline CI/CD & Fluidez \\ \hline
Organizacional & Reducir el número de aprobaciones & Simplicidad \\ \hline
\end{tabular}
\end{center}

Validación de impacto esperado
\begin{enumerate}
    \item Horas recuperadas / mes
    \item Nivel de reducción de IK estimado
    \item Complejidad de implementación (Baja/Media/Alta)
\end{enumerate}

Resultado del Friction RCA
\begin{itemize}
    \item Documenta la causa raíz, su remedio y manual de implementación.
    \item Permite que cada intervención deje aprendizaje institucional, no solo eficiencia temporal.
    \item Crea un repositorio de patrones de fricción recurrentes que pueden preverse en futuros proyectos.
\end{itemize}

\textbf{Valor ejecutivo}
\begin{itemize}
    \item Para líderes técnicos: evita rediseños impulsivos; asegura que cada mejora tiene raíz verificada.
    \item Para negocio: garantiza que los recursos se usan en resolver causas estructurales, no síntomas.
    \item Para cultura: transforma la eliminación de fricciones en un proceso de aprendizaje colectivo.
\end{itemize}

{\epigraph{\textit{El objetivo del Friction RCA no es eliminar trabajo, sino rediseñar flujos para reducir handoffs innecesarios y bloqueos recurrentes.}}{}}

\hrule
% #################################################
% #################################################
\section{Harmony Board}
\begin{quote}
\textit{El tablero de gobernanza ligera.}
\end{quote}

El Harmony Board es el dashboard ejecutivo de Kanso. Consolida las métricas clave de eficiencia operativa (Índice Kanso, horas recuperadas, fricciones resueltas) en una vista mensual para stakeholders de negocio y tecnología.

Basicamente es mostrar de forma clara y ejecutiva, cómo la organización está reduciendo el costo de complejidad a lo largo del tiempo.

\textbf{Objetivo} 
\begin{itemize}
    \item Visibilizar resultados: que la reducción de fricciones se vea en métricas concretas.
    \item Alinear negocio y tecnología: hablar en un mismo lenguaje (horas, dinero, valor recuperado).
    \item Ritualizar la mejora continua: revisar mensualmente para no dejar que la complejidad vuelva a crecer en silencio.
    \item Hacer gobernanza ligera: menos burocracia, más métricas automáticas.
\end{itemize}

\textbf{Modo de uso} 
\begin{enumerate}
    \item Revisión mensual en comité ejecutivo o con líderes de área.
    \item Mostrar avances concretos (fricciones eliminadas, flujos añadidos).
    \item Actualizar (en medida de lo posible) automáticamente datos desde tickets, pipelines y logs.
\end{enumerate}


Estructura de un Harmony Board. Cada entrada debe contener al menos:

\begin{center}
\begin{tabular}{|p{4cm}|l|l|l|l|}
\hline
Indicador & Septiembre & Octubre & Variación & Estado \\ \hline
Índice Kanso (IK) & 22\% & 15\% & -7 pts & Warning \\ \hline
Horas de fricción evitables & 480h & 320h & -33\% & OK \\ \hline
Valor recuperado (USD) & \$24,000 & \$16,000 & -\$8,000 & OK \\ \hline
Friction RCA activos & 3 & 5 & +2 & OK \\ \hline
Top fricción resuelta & \multicolumn{2}{|l|}{CI/CD rotura de credenciales} & Eliminada & OK \\ \hline
\end{tabular}
\end{center}

\textbf{Valor ejecutivo}
\begin{itemize}
    \item Para el CFO: muestra en números el dinero ahorrado y horas recuperadas.
    \item Para el CTO: evidencia qué integraciones y procesos mejoran la eficiencia.
    \item Para los equipos: reconoce logros tangibles (menos horas de fricción).
\end{itemize}

{\epigraph{\textit{El Harmony Board proporciona visibilidad ejecutiva del progreso en reducción de fricción mediante métricas estandarizadas de eficiencia operativa.}}{}}


\hrule
% #################################################
% #################################################
\section{Principios Operativos de Kanso}

El Kanso Framework se fundamenta en cuatro principios que sirven como brújula para guiar decisiones, prácticas y cultura organizacional. Estos principios guían decisiones operativas comunes donde la solución intuitiva (agregar herramientas o procesos) aumenta la complejidad en lugar de reducirla.\\

\textbf{Simplicidad sobre complejidad}
\begin{quote}
\textit{La simplicidad operativa consiste en mantener únicamente los componentes que agregan valor directo al objetivo del sistema.}
\end{quote}

La complejidad suele nace por la necesidad de resolver problemas de forma cada vez más eficaz y eficiente, pero trae consigo el costo inherente a sí misma.

La simplicidad es un acto consciente de diseño: elegir menos, pero mejor.

En Kanso, se priorizan arquitecturas mínimas que hacen lo necesario, evitando adornos innecesarios que después se convierten en fricciones.

Ejemplo:
\begin{itemize}
    \item Un pipeline de despliegue puede tener 15 pasos y 8 aprobaciones manuales. Con Kanso, se redefine para mantener solo los pasos críticos y automatizar el resto, logrando fluidez sin sacrificar control.
\end{itemize}

\textbf{Automatización sobre manualidad}
\begin{quote}
\textit{Lo que puede hacerse de forma automática no debe ser carga para la mente humana.}
\end{quote}

Las tareas repetitivas son caldo de cultivo para errores, dependencia de personas clave y retrasos innecesarios.

Kanso impulsa la automatización como principio rector, no como parche tardío.

El objetivo no es quitar humanidad, sino liberar la atención humana para lo estratégico y creativo.

Ejemplo:
\begin{itemize}
    \item Si una alerta de infraestructura requiere siempre los mismos 4 comandos para remediarse, Kanso lo convierte en un runbook automatizado. El ingeniero se enfoca en las excepciones, no en el trabajo mecánico.
\end{itemize}

\textbf{Estandarización sobre redundancia}
\begin{quote}
\textit{Cada variación no esencial es una deuda que se acumula.}
\end{quote}

Los equipos tienden a resolver problemas con su propia ``receta'': diferentes scripts, lenguajes o frameworks para lo mismo.

Esta diversidad sin propósito genera islas de conocimiento y aumenta el costo de complejidad.

Kanso busca estándares claros: una manera preferida de hacer las cosas, que reduzca el número de opciones y mejore la mantenibilidad.

Ejemplo:
\begin{itemize}
    \item Tres equipos usan diferentes herramientas de infraestructura como código (Terraform, Ansible y scripts bash). Kanso impulsa elegir un estándar común, documentarlo y aplicar gobernanza, reduciendo duplicidad y dependencias.
\end{itemize}

\textbf{Fluidez sobre fricción}
\begin{quote}
\textit{Los procesos deben adaptarse al trabajo, no el trabajo a los procesos.}
\end{quote}

Toda solución técnica introduce overhead operativo. El objetivo es minimizar los puntos de fricción innecesarios que bloquean o ralentizan el flujo de trabajo.

Fluidez significa minimizar los puntos de contacto innecesarios, eliminar cuellos de botella y permitir que la energía se enfoque en la entrega de valor.

En Kanso, cada fricción es vista como una señal: si algo se atasca o requiere esfuerzo desproporcionado, merece rediseño.

Ejemplo:
\begin{itemize}
    \item Un equipo tarda días en desplegar porque cada área (desarrollo, seguridad, operaciones) pide validaciones manuales. Con Kanso, se implementa un flujo automatizado de CI/CD con gates de seguridad embebidos. El resultado es un despliegue fluido y seguro, sin burocracia excesiva.
\end{itemize}

\textbf{Dogma}

Estos principios son heurísticas de decisión, no reglas absolutas. Se aplican según el contexto operativo. El Kanso Framework los utiliza como criterios de decisión pero siempre aplicado en un contexto y no tomarse de manera literal:
\begin{itemize}
    \item Ante dos caminos, se elige el más simple.
    \item Ante una tarea repetitiva, se automatiza.
    \item Ante múltiples formas de hacer lo mismo, se estandariza.
    \item Ante un proceso con trabas, se rediseña para lograr fluidez.
\end{itemize}

De esta forma, Kanso convierte la reducción del costo de complejidad en una filosofía práctica, aplicable tanto en tecnología como en cualquier operación humana.\\

\textbf{Modo de uso}
\begin{itemize}
    \item Tratar de reducir en la medida de lo posible una fricción. ¿Qué necesita esta fricción para reducir su impacto? Simplificar, Automatizar, Estandarizar o Fluir.
\end{itemize}


\textbf{Valor ejecutivo}
\begin{itemize}
    \item Para líderes: convierten la filosofía en reglas aplicables y rápidas de evaluar.
    \item Para equipos: simplifican discusiones; no se debate en abstracto, se contrasta con reglas claras.
    \item Para el framework: hacen que Kanso no sea solo teoría, sino algo práctico en la medida de lo posible.
\end{itemize}

{\epigraph{\textit{Los Principios Operativos de Kanso funcionan como criterios de decisión: ante dos alternativas, se elige la que reduce fricción medible.}}{}}


\hrule
% #################################################
% #################################################
\section{Checklist de Clasificación de Fricciones}
\begin{quote}
\textit{No toda fricción puede ni debe eliminarse.}
\end{quote}

\textbf{Objetivo} 
\begin{itemize}
    \item Distinguir entre fricción inevitable y fricción evitable, asegurando que Kanso actúe solo donde el esfuerzo tiene retorno.
\end{itemize}


\textbf{Fricciones Inevitables (aceptar, contener o automatizar parcialmente)}

Si se cumplen 3 o más de los siguientes criterios, la fricción se clasifica como inevitable:

\begin{center}
\begin{tabular}{|l|p{5cm}|p{6cm}|}
\hline
\# & Criterio & Ejemplo típico \\ \hline
1 & Regulatoria o de compliance. & Validaciones SOX, PCI DSS, GDPR, etc. \\ \hline
2 & Dependencia física o contractual. & Tiempos de proveedor o hardware externo. \\ \hline
3 & Restricción técnica universal. & Latencia geográfica, límites de protocolo. \\ \hline
4 & Costo de mitigación > beneficio esperado. & Automatización que costaría más que el ahorro. \\ \hline
5 & Riesgo operativo alto si se elimina. & Control manual de seguridad que evita incidentes críticos. \\ \hline
\end{tabular}
\end{center}

Tratamiento: documentar, contener, y optimizar su frecuencia, no eliminarla.\\

\textbf{Fricciones Evitables (priorizar en Friction RCA)}\\

Si se cumplen 2 o más de los siguientes criterios, la fricción se clasifica como evitable:

\begin{center}
\begin{tabular}{|l|p{5cm}|p{6cm}|}
\hline
\# & Criterio & Ejemplo típico \\ \hline
1 & Repetitiva y/o manual. & Creación manual de tickets o reportes. \\ \hline
2 & Redundante o duplicada. & Flujos paralelos que realizan la misma acción. \\ \hline
3 & Falta de estandarización. & Configuraciones diferentes por equipo o entorno. \\ \hline
4 & Error recurrente. & Deploys fallidos por pasos manuales. \\ \hline
5 & Burocrática o sin valor añadido. & Aprobaciones jerárquicas que no cambian resultados. \\ \hline
\end{tabular}
\end{center}

Tratamiento: priorizar en el Friction Ledger, ejecutar un Friction RCA y documentar su implementación.\\

\textbf{Uso Práctico}

\begin{center}
\begin{tabular}{|l|p{6cm}|p{4cm}|}
\hline
Paso & Acción & Artefacto \\ \hline
1 & Registrar la fricción detectada & Friction Ledger \\ \hline
2 & Evaluar criterios inevitables/evitables & Checklist \\ \hline
3 & Clasificar inevitable o evitable & Friction Ledger \\ \hline
4 & Si evitable: lanzar Friction RCA y documentar \\ \hline
5 & Si inevitable: documentar y contener & Harmony Board \\ \hline
\end{tabular}
\end{center}
