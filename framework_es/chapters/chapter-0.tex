\chapter*{Abstract}\label{chap:zero}
\addcontentsline{toc}{chapter}{Resumen}
\textsl{
Las organizaciones de TI modernas operan ecosistemas fragmentados de más de 50 herramientas, servicios y procesos. 
Si bien cada componente promete eficiencia, su integración genera gastos operativos ocultos: traspasos manuales, 
flujos de trabajo redundantes y fallos recurrentes. Estos gastos, que a menudo representan entre el  15\% y 40\% del gasto operativo total, 
no se miden con los modelos tradicionales de coste total de propiedad.
Kanso Framework proporciona (IK), una métrica que mide qué porcentaje del gasto operativo se desperdicia en ineficiencias evitables. 
A través de herramientas estructuradas como el Friction Ledger, Friction RCA y Harmony Board, 
los equipos pueden identificar las causas raíz, priorizar las intervenciones y transformar 
la fricción en fluidez operativa.
\footnote{
\textit{Declaración de autoría:}  
El Kanso Framework surge de la experiencia personal y profesional de Eduardo López.  
Las ideas aquí presentadas buscan contribuir a la reflexión y práctica sobre la simplicidad operativa, reconociendo la inspiración de marcos abiertos como Scrum, ITIL y Lean.}
\\
}

\textbf{Palabras clave:} Fricción operativa, OPEX oculto, Simplicidad operativa, Gobernanza ligera.

% \begin{table}[h]
% \centering
% \renewcommand{\arraystretch}{1.3}
% \begin{tabular}{|p{3cm}|p{4cm}|p{6cm}|}
% \hline
% \textbf{Versión} & \textbf{Fecha} & \textbf{Revisado por} \\ \hline
% 0.9 & 23 de octubre de 2025 & Eduardo López \\ \hline
% 1.0 & 24 de octubre de 2025 & Joel Chan, Eduardo López, Aaron Vega\\ \hline
% \multicolumn{3}{|l|}{\textit{Notas:}} \\[0.3em]
% \multicolumn{3}{|p{13cm}|}{
% Este documento forma parte del desarrollo original del \textbf{Kanso Framework}.  
% Cualquier modificación posterior deberá registrarse en esta tabla con su respectiva fecha y responsable de revisión.} \\ \hline
% \end{tabular}
% \end{table}