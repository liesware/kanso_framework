\chapter{La Implementación}\label{chap:three}

% #################################################
% #################################################
\section{Roles y Responsabilidades}

\textbf{Objetivo}\\
Definir roles y responsabilidades dentro del framework para tener claridad. Kanso utiliza 3 roles que evolucionan según la fase de adopción, manteniendo la simplicidad operativa como principio rector.
\\
\\
\textbf{Filosofía de Roles en Kanso}

\textit{``Los roles no son puestos permanentes, son funciones que emergen según el contexto.''}

\begin{itemize}
    \item Los roles se activan progresivamente según madurez
    \item El ownership es compartido y rotativo
    \item La complejidad de roles crece solo si agrega valor
    \item Cada rol tiene fronteras claras para evitar confusión
\end{itemize}

Resumen de Roles por Fase

\begin{center}
\small
\begin{tabular}{|l|c|c|c|}
\hline
\textbf{Rol} & \textbf{Kanso Lite} & \textbf{Kanso Core} & \textbf{Kanso Full} \\ \hline
Kanso Lead & Formal & Formal & Formal \\ \hline
Friction Resolvers & Informal & Semi-formal & Formal \\ \hline
Business Link & No existe & Semi-formal & Formal \\ \hline
\end{tabular}
\end{center}

\hrule
% #################################################
% #################################################
\section{Kanso Lead: El Facilitador y Documentador}

Naturaleza del rol: Detective + Facilitador de Información + Archivista\\

Kanso Lead: Rol dedicado a la detección, documentación y facilitación de la reducción de fricciones.
Equivalencias según contexto organizacional:

\begin{itemize}
    \item En equipos ágiles: puede ser el Scrum Master o Agile Coach
    \item En equipos de producto: puede ser un PM con enfoque en eficiencia
    \item En equipos técnicos: puede ser un Staff Engineer o SRE con ownership de operaciones
    \item En organizaciones maduras: puede ser un rol dedicado (Operational Excellence Lead)
\end{itemize}

\textbf{Responsabilidades Principales}
\\
\\
Detección Proactiva de Fricciones
\begin{itemize}
    \item Observa procesos, métricas y quejas recurrentes
    \item Identifica patrones de fricción antes de que escalen
    \item No espera reportes pasivos, busca activamente
    \item Realiza entrevistas breves (15min) para validar fricciones
\end{itemize}

Facilitación de Friction RCA Colaborativo
\begin{itemize}
    \item Convoca y modera sesiones Friction RCA con Friction Resolvers
    \item Hace preguntas causales: ``¿Por qué ocurre esto?'' (3 whys)
    \item No propone soluciones, facilita que el equipo las encuentre
    \item Documenta causas raíz y contramedidas propuestas
\end{itemize}

Documentación de Implementación
\begin{itemize}
    \item Registra Friction RCA completo en Friction Ledger
    \item Trabaja con el Friction Resolvers para documentar ``Pasos para Reaplicar''
    \item Crea mini-runbook cuando la solución es reutilizable
    \item Asegura que criterios de éxito estén claros
\end{itemize}

Gestión del Índice de Friction RCA
\begin{itemize}
    \item Mantiene índice ordenado de Friction RCAs cerrados
    \item Añade tags para facilitar búsqueda futura
    \item Registra reaplicaciones cuando un Friction RCA se reutiliza
    \item Tiempo: 2-3 minutos por Friction RCA cerrado
\end{itemize}

Cálculo y Reporte de IK
\begin{itemize}
    \item Calcula Índice Kanso mensualmente
    \item Actualiza Harmony Board con métricas clave
    \item Presenta resultados a Business Link o liderazgo
    \item Identifica tendencias y fricciones críticas
\end{itemize}

Facilitación de Ceremonias
\begin{itemize}
    \item Friction Review (quincenal): revisión de fricciones y quick wins
    \item Efficiency Sync (menusual): presentación de IK y priorización
    \item Strategic Review (trimestral): estrategia y ajuste de roadmap (en Full)
\end{itemize}

Entregables
\begin{itemize}
    \item Friction Ledger actualizado semanalmente
    \item Friction Analysis Matrix Board con análisis causales completos
    \item Documentación de Implementación con pasos de reaplicación
    \item Índice de Friction RCAs ordenado y mantenido
    \item IK Dashboard con métricas mensuales
    \item Harmony Board (en fases Core/Full)
    \item Reportes ejecutivos mensuales
\end{itemize}

Decide:
\begin{itemize}
    \item Cómo facilitar sesiones Friction RCA
    \item Qué fricciones documentar (threshold: >2h/mes)
    \item Formato y nivel de detalle en documentación
    \item Frecuencia de ceremonias (dentro de rangos sugeridos)
\end{itemize}

No decide:
\begin{itemize}
    \item Qué fricciones resolver (eso es Business Link)
    \item Cómo implementar soluciones (eso es Friction Resolvers)
    \item Priorización estratégica (eso es Business Link)
    \item Recursos asignados (eso es Business Link)
\end{itemize}

NO es su Rol
\begin{itemize}
    \item No ejecuta soluciones técnicas ni operativas
    \item No propone contramedidas (las facilita, no las crea)
    \item No es dueño de fricciones específicas
    \item No diseña implementaciones técnicas
    \item No aprueba recursos ni presupuestos
\end{itemize}

Perfil Ideal
\begin{itemize}
    \item Experiencia en facilitación o Scrum Master
    \item Pensamiento sistémico y curiosidad por procesos
    \item Habilidad para documentar claramente
    \item No necesita ser técnico, pero debe entender el contexto
    \item Capacidad de hacer preguntas sin juzgar
\end{itemize}

Dedicación por Fase
\begin{center}
\small
\begin{tabular}{|l|c|l|}
\hline
\textbf{Fase} & \textbf{Horas/semana} & \textbf{Puede ser} \\ \hline
Kanso Lite & 2-4h & Part-time, rotativo \\ \hline
Kanso Core & 4-6h & Part-time formal \\ \hline
Kanso Full & 6-8h & Part-time o Full-time (50\%) \\ \hline
\end{tabular}
\end{center}

Métricas de Éxito del Rol
\begin{itemize}
    \item Friction Ledger actualizado >90\% del tiempo
    \item Friction RCAs cerrados tienen ``Pasos para Reaplicar'' en >70\% de casos complejos
    \item Índice de Friction RCAs consultado antes de crear nuevo Friction RCA en >60\% de casos
    \item IK calculado y reportado el mismo día cada mes
    \item Ceremonias inician y terminan a tiempo
\end{itemize}

\hrule
% #################################################
% #################################################
\section{Friction Resolvers: Equipo Operativo}

Naturaleza del rol: Ejecutores + Diseñadores + Documentadores de Conocimiento

Filosofía Core

\begin{quote}
\textit{``Los que hacen el trabajo proponen la solución y documentan el conocimiento.''}
\end{quote}

Los Friction Resolvers son los miembros del equipo operativo que:

\begin{itemize}
    \item Reportan fricciones cuando las encuentran
    \item Participan en el análisis de causa raíz
    \item Implementan las contramedidas aprobadas
    \item Documentan soluciones para reaplicación
\end{itemize}

Friction Resolvers no es un rol jerárquico, es una función colectiva del equipo operativo. Todos contribuyen según capacidad, contexto y disponibilidad. El ownership es rotativo y emerge naturalmente.\\
\\
\textbf{Responsabilidades Principales}
\\
\\
Reporte de Fricciones (Detección)
\begin{itemize}
    \item Reportan fricciones cuando las encuentran en su trabajo diario
    \item Proveen evidencia: tickets, logs, tiempos medidos
    \item No necesitan analizar causa raíz (eso se hace en Friction RCA colaborativo)
    \item Formato simple: ``Qué ocurre, cuándo, impacto estimado''
\end{itemize}

Búsqueda de Soluciones Existentes
\begin{itemize}
    \item Antes de proponer nueva solución, consultan el índice de Friction RCAs
    \item Preguntan: ``¿Alguien ya resolvió esto?''
    \item Si existe Friction RCA similar, lo reutilizan (adaptándolo si es necesario)
    \item Registran reaplicación en Friction RCA original
\end{itemize}

Participación en Friction RCA Colaborativo
\begin{itemize}
    \item Asisten a sesiones Friction RCA facilitadas por el Kanso Lead
    \item Proponen contramedidas basadas en experiencia técnica
    \item Discuten viabilidad, esfuerzo y riesgos
    \item Co-crean análisis causal con el equipo
\end{itemize}

Implementación de Soluciones
\begin{itemize}
    \item Ejecutan contramedidas priorizadas por Business Link
    \item Trabajan en soluciones durante ciclos normales de trabajo (sprints, etc.)
    \item Validan que la fricción efectivamente se redujo
    \item Reportan progreso a Kanso Lead para actualización del Ledger
\end{itemize}

Documentación de Implementación
\begin{itemize}
    \item Para soluciones complejas (>8h trabajo): documentan ``Pasos para Reaplicar''
    \item Colaboran con el Kanso Lead para crear mini-runbook
    \item Formato: 5-10 bullets, lenguaje claro, sin jerga innecesaria
    \item Incluyen criterios de éxito: ``¿Cómo saber que funcionó?''
\end{itemize}

Ejemplo de documentación Friction Resolvers:
\begin{quote}
\textit{Pasos para Reaplicar (FRCA-001: CI/CD credenciales):}
\begin{enumerate}
    \item Instalar HashiCorp Vault agent en runners de CI/CD
    \item Configurar dynamic secrets con TTL de 24h
    \item Modificar pipeline YAML: reemplazar env vars por \texttt{vault:read}
    \item Ejecutar test deploy en ambiente staging
    \item Validar: zero fallos por credenciales en 2 semanas
\end{enumerate}
\end{quote}

Ownership Temporal de Fricciones
\begin{itemize}
    \item En fases Core/Full, miembros de Friction Resolvers se auto-asignan como owners temporales
    \item El Ownership dura solo hasta que la fricción se resuelve
    \item No hay owners permanentes de áreas (para evitar silos)
    \item Principio: ``El que mejor entiende la fricción, la resuelve''
\end{itemize}

Entregables
\begin{itemize}
    \item Reportes de fricciones con evidencia
    \item Contramedidas propuestas en Friction RCA colaborativo
    \item Soluciones implementadas y validadas
    \item Sección ``Pasos para Reaplicar'' en Friction RCAs complejos (colaboración con el Kanso Lead)
    \item Criterios de éxito documentados
    \item Registro de reaplicaciones cuando reutilizan Friction RCA existente
    \item Actualizaciones de progreso en el Friction Ledger
\end{itemize}

Decide:
\begin{itemize}
    \item Qué contramedida proponer (basado en experiencia técnica)
    \item Cómo implementar la solución (autonomía técnica)
    \item Nivel de detalle en documentación de pasos
    \item Cuándo una solución es suficientemente genérica para reaplicar
    \item Auto-asignación como owner temporal de fricción
\end{itemize}

NO decide:
\begin{itemize}
    \item Qué fricciones priorizar (eso es Business Link)
    \item Recursos disponibles o deadlines (eso es Business Link)
    \item Formato de artefactos Kanso (eso es Kanso Lead)
\end{itemize}

NO es su Rol
\begin{itemize}
    \item No facilita ceremonias Kanso (eso es Kanso Lead)
    \item No mantiene Friction Ledger ni Índice de Friction RCAs (eso es Kanso Lead)
    \item No prioriza estratégicamente (eso es Business Link)
    \item No reporta a ejecutivos (eso es Kanso Lead)
    \item No son owners permanentes de áreas completas
\end{itemize}

Perfil Ideal
\begin{itemize}
    \item Miembros regulares del equipo operativo (devs, SREs, PMs, analistas)
    \item Conocimiento técnico/operativo del dominio
    \item Disposición a documentar soluciones (mentalidad de compartir conocimiento)
    \item Capacidad de proponer soluciones pragmáticas
    \item No necesitan ser seniors, pero sí tener ownership
\end{itemize}

\textbf{Evolución del Rol por Fase}

En Kanso Lite (informal):
\begin{itemize}
    \item Todos en el equipo son Friction Resolvers sin rol formal
    \item Contribuyen cuando tienen tiempo/contexto
    \item No hay asignaciones ni ceremonias adicionales
    \item Documentación mínima (solo si la solución se repetirá)
\end{itemize}

En Kanso Core (semi-formal):
\begin{itemize}
    \item Friction Resolvers se auto-organiza por tipo de fricción
    \item Aparecen owners temporales naturalmente
    \item La Documentación se vuelve esencial para soluciones reutilizables
    \item Participan activamente en Friction RCA colaborativos
\end{itemize}

En Kanso Full (formal):
\begin{itemize}
    \item Ownership claro por fricción (auto-asignado o negociado)
    \item Lideran Friction RCA sessions de fricciones en su área de expertise
    \item Mantienen documentación de implementación actualizado si cambia el contexto
    \item Mentoría a otros Friction Resolvers en reaplicación de soluciones
\end{itemize}

\textit{Nota: Estas horas se distribuyen entre todo el Friction Resolvers, no por persona.}

Dedicación por Fase
\begin{center}
\small
\begin{tabular}{|l|c|l|}
\hline
\textbf{Fase} & \textbf{Horas/semana/persona} & \textbf{Naturaleza} \\ \hline
Kanso Lite & 30min & Participación en Sync, reporte ad-hoc \\ \hline
Kanso Core & 2-3h & Implementación + documentación \\ \hline
Kanso Full & 4-6h & Ownership formal + mentoría \\ \hline
\end{tabular}
\end{center}

Métricas de Éxito del Rol

\begin{itemize}
    \item >70\% de soluciones implementadas reducen la fricción según validación
    \item >70\% de Friction RCAs complejos tienen ``Pasos para Reaplicar'' documentados
    \item >60\% de nuevas fricciones similares reutilizan Friction RCA existente
    \item Tiempo de reaplicación <25\% del tiempo de solución original
    \item Satisfacción del equipo con claridad de soluciones documentadas
\end{itemize}

\hrule
% #################################################
% #################################################
\section{Business Link: Sponsor Ejecutivo}

Naturaleza del rol: Puente Negocio-Tecnología + Asignador de Recursos

Filosofía Core

\begin{quote}
\textit{``Aparece cuando el equipo necesita dirección estratégica, no supervisión operativa.''}
\end{quote}

El Business Link es el stakeholder de negocio que prioriza qué fricciones resolver según impacto estratégico 
y disponibilidad de recursos. Equivalencias por contexto:

\begin{itemize}
    \item En startups: típicamente el CTO o VP Engineering
    \item En product teams: Product Director o Head of Engineering
    \item En enterprise IT: IT Operations Manager o Director of Engineering Excellence
\end{itemize}

\textbf{Responsabilidades Principales}
\\
\\
Priorización de Fricciones
\begin{itemize}
    \item Recibe reporte del Kanso Lead con fricciones y Friction RCAs propuestos
    \item Evalúa impacto de negocio vs. esfuerzo de implementación
    \item Decide qué fricciones atacar primero, cuáles ir a backlog
    \item Considera estrategia de negocio, urgencia y recursos disponibles
\end{itemize}

Aprobación de Recursos
\begin{itemize}
    \item Aprueba tiempo del Friction Resolvers dedicado a reducción de fricciones
    \item Puede autorizar recursos adicionales (contrataciones, herramientas)
    \item Negocia trade-offs con otros stakeholders si hay conflictos
    \item Protege al equipo de sobrecarga (balance entre delivery y eficiencia)
\end{itemize}

Alineación con Estrategia de Negocio
\begin{itemize}
    \item Conecta reducción de fricciones con OKRs/KPIs del negocio
    \item Traduce IK en métricas que importan a ejecutivos (ROI, time-to-market, etc.)
    \item Comunica valor de Kanso a otros líderes
    \item Ajusta prioridades si cambia estrategia organizacional
\end{itemize}

Revisión de Resultados
\begin{itemize}
    \item Revisa el Harmony Board mensualmente (en Core/Full)
    \item Valida que inversión en reducción de fricciones genera el ROI esperado
    \item Ajusta enfoque si el IK no mejora o si hay mejor uso de recursos
    \item Celebra wins con el equipo (reconocimiento de mejoras)
\end{itemize}

Entregables
\begin{itemize}
    \item Decisión de priorización de fricciones (mensual)
    \item Aprobación de recursos asignados a reducción de complejidad
    \item Alineación de Kanso con estrategia de negocio (trimestral en Full)
    \item Feedback sobre ROI de iniciativas completadas
    \item Comunicación de valor de Kanso a stakeholders externos
\end{itemize}

Decide:
\begin{itemize}
    \item Qué fricciones se priorizan (basado en impacto de negocio)
    \item Cuánto tiempo de Friction Resolvers se dedica a Kanso vs. delivery
    \item Aprobación de recursos (herramientas, contrataciones si aplica)
    \item Ajuste de enfoque estratégico de Kanso
\end{itemize}

NO decide:
\begin{itemize}
    \item Cómo se implementan soluciones (eso es el Friction Resolvers)
    \item Cómo se facilita el proceso (eso es el Kanso Lead)
    \item Qué fricciones documentar (eso es el Kanso Lead)
    \item Detalles técnicos de Friction RCAs (eso es el Friction Resolvers)
\end{itemize}

NO es su Rol
\begin{itemize}
    \item No facilita ceremonias ni Friction RCA sessions (eso es el Kanso Lead)
    \item No ejecuta soluciones (eso es el Friction Resolvers)
    \item No diseña contramedidas técnicas (eso es el Friction Resolvers)
    \item No mantiene artefactos Kanso (eso es el Kanso Lead)
    \item No es un supervisor operativo del día a día
\end{itemize}

Perfil Ideal
\begin{itemize}
    \item Líder con visión de negocio y técnica (CTO, VP Engineering, Product Director)
    \item Capacidad de priorizar con criterio de ROI
    \item Empodera equipos en lugar de microgestionar
    \item Entiende trade-offs entre eficiencia operativa y delivery
    \item Comunicador efectivo con stakeholders de negocio
\end{itemize}


\textbf{Evolución del Rol por Fase}

En Kanso Lite:
\begin{itemize}
    \item \textbf{No existe formalmente}
    \item El mismo equipo prioriza por sentido común
    \item Kanso Lead consulta informalmente con líder técnico si hay duda
\end{itemize}

En Kanso Core:
\begin{itemize}
    \item \textbf{Sombrero temporal}
    \item Aparece en Monthly Reviews (1h/mes)
    \item Típicamente: Product Owner, Engineering Manager, CTO, VP of Ops
    \item Durante la sesión, usa el ``sombrero'' de Business Link
    \item Fuera de la sesión, vuelve a su rol normal
\end{itemize}

En Kanso Full:
\begin{itemize}
    \item Rol formal
    \item Participa activamente en Harmony Sessions y Quarterly Reviews
    \item Puede ser un comité en organizaciones grandes (Product + Engineering + Ops)
    \item Dedicación: 2-4h/mes
\end{itemize}

Dedicación por Fase

\begin{center}
\small
\begin{tabular}{|l|c|l|}
\hline
\textbf{Fase} & \textbf{Horas/mes} & \textbf{Naturaleza} \\ \hline
Kanso Lite & 0h & No existe \\ \hline
Kanso Core & 1-2h & Sombrero temporal (Monthly Review) \\ \hline
Kanso Full & 2-4h & Rol formal (Harmony + Quarterly) \\ \hline
\end{tabular}
\end{center}

Métricas de Éxito del Rol

\begin{itemize}
    \item Decisiones de priorización tomadas en <30min por sesión
    \item >70\% de fricciones priorizadas se implementan dentro del ciclo acordado
    \item IK mejora trimestre a trimestre (validando la efectividad de la priorización)
    \item Equipo siente que sus propuestas son escuchadas y valoradas
    \item ROI de iniciativas Kanso es comunicado claramente a los stakeholders
\end{itemize}

\section{Comparativa de Roles}

\begin{center}
\scriptsize
\begin{tabular}{|p{2.5cm}|p{3.5cm}|p{3.5cm}|p{3.5cm}|}
\hline
\textbf{Dimensión} & \textbf{Kanso Lead} & \textbf{Friction Resolvers} & \textbf{Business Link} \\ \hline
\textbf{Función principal} & Facilita y documenta & Propone y ejecuta & Prioriza y aprueba \\ \hline
\textbf{Poder de decisión} & Cómo facilitar el proceso & Cómo implementar soluciones & Qué priorizar y recursos \\ \hline
\textbf{Dedicación (Core)} & 4-6h/semana & 2-3h/semana/persona & 1-2h/mes \\ \hline
\textbf{Artefactos clave} & Friction Ledger, Friction Analysis Matrix, Índice & Documentación de Implementación(pasos), Soluciones & Decisiones de priorización \\ \hline
\textbf{Interacción con otros} & Facilita a todos & Colabora con Lead, reporta a Link & Recibe info de Lead, dirige a Friction Resolvers \\ \hline
\textbf{NO hace} & Ejecutar, proponer, priorizar & Facilitar, mantener ledger, aprobar & Facilitar, ejecutar, documentar \\ \hline
\textbf{Señal de éxito} & Artefactos actualizados,Friction RCAs claros & Soluciones implementadas, documentadas & IK mejorando, equipo empoderado \\ \hline
\end{tabular}
\end{center}

\textbf{Principios de Colaboración entre Roles}

\begin{enumerate}
    \item El Kanso Lead facilita, no controla: Ayuda al flujo de información, no impone decisiones.
    \item El Friction Resolvers propone desde experiencia: Las mejores soluciones vienen de quienes hacen el trabajo.
    \item El Business Link aparece cuando agrega valor: No supervisa día a día, solo dirige estratégicamente.
    \item El Ownership es compartido: Todos son responsables de reducir fricciones, no solo Kanso Lead.
    \item La Documentación es colaborativa: Kanso Lead escribe, Friction Resolvers valida y complementa.
    \item Las Decisiones por consenso en Friction RCA: Business Link prioriza, pero Friction Resolvers puede escalar si hay desacuerdo razonable.
\end{enumerate}

\textbf{Anti-Patrones a Evitar}

\begin{itemize}
    \item Kanso Lead como cuello de botella: Si todo pasa por una persona, el framework falla. Distribuir ownership.
    \item Friction Resolvers sin autonomía: Si no pueden proponer soluciones, se vuelven ejecutores pasivos. Empoderar.
    \item Business Link como microgerente: Si revisa cada fricción diariamente, genera overhead. Confiar y delegar.
    \item Roles permanentes y rígidos: Si nadie puede rotar, se crean silos. Fomentar flexibilidad.
    \item Documentación como castigo: Si documentar se siente burocrático, nadie lo hará. Mantener simple y útil.
\end{itemize}

\textbf{Preguntas Frecuentes sobre Roles}
\\ 
\\ 
\textit{¿Puede una persona tener múltiples roles?}\\ 
Sí, especialmente en equipos pequeños. Un Tech Lead puede ser Kanso Lead + Business Link temporalmente. Lo importante es separar las funciones mentalmente.

\textit{¿Qué pasa si nadie quiere ser Kanso Lead?}\\ 
Rota el rol cada trimestre. 4-6h/semana no es una carga grande si se distribuye. Alternativamente, un PM o Scrum Master puede asumir naturalmente.

\textit{¿Friction Resolvers debe documentar TODAS las soluciones?}\\ 
No. Solo soluciones complejas (>8h trabajo) y reutilizables. Quick wins triviales no necesitan documentación extensa.

\textit{¿Business Link puede ser un comité?}\\ 
Sí, en organizaciones grandes. Ej: Product Director + Engineering Manager + Head of Ops. Lo importante es que decidan rápido (no por consenso eterno).

\textit{¿Qué pasa si Business Link nunca tiene tiempo?}\\ 
Red flag. Significa que Kanso no tiene sponsor ejecutivo. Considera escalar o implementar solo Kanso Lite (sin Business Link formal) hasta tener buy-in.
\\ 
\hrule
% #################################################
% #################################################
\section{Unidad de Trabajo}


\textbf{Objetivo}\\
Detectar, clasificar, cuantificar y priorizar las fricciones operativas del equipo mediante un ciclo estructurado de observación, análisis y corrección.

\textbf{Duración}\\
2 semanas (recomendado) — Este período representa el equilibrio óptimo entre acción continua y análisis reflexivo, permitiendo capturar patrones significativos sin extender excesivamente el tiempo entre correcciones.
\\ 
\\ 
\textbf{Naturaleza del Ciclo de Trabajo}
\\ 
El ciclo de trabajo constituye la unidad fundamental de mejora continua en Kanso. A diferencia de metodologías tradicionales centradas en entregas o cumplimiento de tareas, este ciclo opera bajo los siguientes principios:

\begin{itemize}
    \item Foco en flujo, no en output: El ciclo no mide entregas completadas ni tareas cerradas, sino la calidad del flujo operativo y la reducción sistemática de fricción.
    \item Introspección estructurada: Funciona como el espacio designado donde la organización observa, mide y corrige su eficiencia interna de manera metódica y basada en evidencia.
    \item Sincronización con operación: El ciclo de trabajo comienza y termina alineado con el ciclo operativo natural del equipo, respetando sus ritmos de entrega y planificación.
    \item Responsabilidad del Kanso Lead: Este ciclo es exclusivo del Kanso Lead, quien lo utiliza para evaluar el estado del sistema, planear intervenciones y presentar las cargas de trabajo optimizadas antes del inicio del siguiente ciclo operativo del equipo.
\end{itemize}

\textbf{Estructura del Ciclo de Trabajo}

El ciclo de trabajo se organiza en actividades distribuidas estratégicamente a lo largo de dos semanas, cada una con entregables específicos que alimentan el proceso de mejora continua:

\begin{center}
\small
\begin{tabular}{|p{3cm}|p{5cm}|p{5cm}|}
\hline
\textbf{Día/Período} & \textbf{Actividad} & \textbf{Artefactos / Resultados} \\ \hline
Durante todo el ciclo & Detección de fricciones mediante entrevistas informales, análisis de logs, revisión de tickets y monitoreo de pipelines. & Llenado progresivo del \textit{Friction Ledger} con eventos de fricción categorizados y cuantificados. \\ \hline
Durante todo el ciclo  & Análisis Friction RCA: evaluación de Riesgo, Frecuencia y Alcance de cada fricción detectada. & Llenado y actualización del \textit{Friction Analysis Matrix} con puntuaciones y prioridades calculadas. \\ \hline
A la mitad del ciclo & Sesión de revisión del \textit{Friction Analysis Matrix}: validación de prioridades y aprobación de reordenamientos. & Decisiones documentadas sobre qué fricciones atacar en el siguiente ciclo operativo del equipo. \\ \hline
Al final del ciclo & Actualización de métricas del sistema: recálculo del Índice Kanso y otras métricas derivadas. & Nuevo IK parcial, variación porcentual documentada, tendencias identificadas. \\ \hline
Cierre del ciclo  & Generación de reporte de resultados del ciclo completo. & Minireporte visual o entrada automática en el \textit{Harmony Board} con resumen ejecutivo de mejoras implementadas. \\ \hline
\end{tabular}
\end{center}

\textbf{Criterios de Éxito}

Un ciclo de trabajo se considera exitoso cuando cumple con los siguientes criterios mínimos:

\begin{itemize}
    \item Al menos 50\% de las fricciones detectadas están documentadas en el \textit{Friction Ledger}.
    \item El \textit{Friction Analysis Matrix} contiene análisis cuantitativo completo para todas las fricciones prioritarias.
    \item Se han aprobado al menos 3 acciones concretas de reducción de fricción para el siguiente ciclo.
    \item El IK muestra variación medible (positiva o negativa) respecto al ciclo anterior.
    \item El reporte de cierre está disponible para consulta del equipo antes del inicio del siguiente ciclo operativo.
\end{itemize}



\hrule
% #################################################
% #################################################
\section{Artefactos Clave}

\begin{center}
\begin{tabular}{|p{3.5cm}|p{4.5cm}|p{5cm}|}
\hline
Artefacto & Propósito & Contenido principal \\ \hline
Friction Ledger & Registro vivo y cuantificable de fricciones. & ID, descripción, costo, evidencia, acción sugerida. \\ \hline
Friction Analysis Matrix & Diagnóstico causal y reordenamiento empírico. & Causas raíz, contramedidas Kanso, impacto estimado. \\ \hline
Harmony Board & Visión mensual y evolutiva del IK. & Indicadores, variaciones, estado de cada flujo. \\ \hline
\end{tabular}
\end{center}

\textbf{Friction Ledger} \\
Registro centralizado, vivo y cuantificable.

Columnas básicas:

\begin{center}
\scriptsize
\begin{tabular}{|l|p{2.5cm}|l|l|l|l|p{2cm}|p{2cm}|l|}
\hline
ID & Descripción & Clasificación & Horas/mes & Costo hora & Costo mensual & Evidencia & Acción sugerida & Estado \\ \hline
F-001 & Integración CI/CD rompe credenciales & Evitable & 40 & 50 & 2,000 & Logs, Jira & Automatizar con Vault & Cerrada \\ \hline
F-002 & Auditoría PCI trimestral & Inevitable & 20 & 70 & 1,400 & Requerimiento legal & Mantener & Abierta \\ \hline
\end{tabular}
\end{center}

\textbf{Friction Analysis Matrix} \\
Análisis de causas.

Columnas básicas:

\begin{center}
\scriptsize
\begin{tabular}{|l|p{2.5cm}|p{2.5cm}|l|p{2.5cm}|p{2cm}|}
\hline
\# & Causa raíz & Descripción / Evidencia & Nivel & Contramedida Kanso & Regla aplicable \\ \hline
1 & Rotación de credenciales manual & Logs de fallos que muestran expiración recurrente. & Técnica & Automatizar rotación con Vault. & Automatización \\ \hline
2 & Falta de ownership del secret store & Entrevistas con DevOps confirman ambigüedad de responsables. & Organizacional & Definir responsable de gestión de secretos. & Estandarización \\ \hline
3 & Política de rotación no automatizada ni auditada & Documento de seguridad sin evidencia de pipeline integrado. & Proceso & Integrar rotación y auditoría en pipeline CI/CD. & Fluidez. \\ \hline
4 & Demasiadas pasos para aprobar un despliegue & Tiempos de respuesta & Proceso & Reducir Burocracia & Simplicidad \\ \hline
\end{tabular}
\end{center}


\textbf{Harmony Board} \\ 
Tablero mensual de control.

Columnas básicas:

\begin{center}
\begin{tabular}{|p{3.5cm}|l|l|l|l|}
\hline
Indicador & Sept & Oct & Variación & Estado \\ \hline
IK & 22\% & 15\% & -7 pts & OK \\ \hline
Horas evitables & 480 & 320 & -33\% & OK \\ \hline
Valor recuperado & \$24,000 & \$16,000 & -8k & OK \\ \hline
Friction RCA Cerradas & 3 & 5 & +2 & OK \\ \hline
\end{tabular}
\end{center}

\textbf{Kanso sessions} \\ 
Reuniones para revisar estados del framework.

Columnas básicas:

\begin{center}
\small
\begin{tabular}{||p{2.5cm}|p{2.5cm}|p{2cm}|p{5.5cm}||}
\hline
\textbf{Participantes} & \textbf{Nombre} & \textbf{Frecuencia} & \textbf{Objetivo} \\ \hline
Kanso Lead + Friction Resolvers & Friction Review & Semana 2 de cada ciclo & Revisar Friction Ledger, priorizar Friction RCA. \\ \hline
Kanso Lead + Business Link + Tech Leads & Efficiency Sync & Mensual & Presentar IK, validar ROI de mejoras, ajustar prioridades. \\ \hline
Leadership team completo & Strategic Review & Cada 3 meses & Evaluar tendencias, ajustar estrategia, aprobar inversiones mayores. \\ \hline
\end{tabular}
\end{center}



\hrule
% #################################################
% #################################################
\section{Sistema de Madurez Kanso}

\textbf{Objetivo}\\
Evaluar el progreso cultural y operativo de la organización.

\begin{center}
\begin{tabular}{|l|p{3cm}|p{6cm}|p{2cm}|}
\hline
Nivel & Nombre & Características & IK Promedio \\ \hline
1 & Caótico & No hay visibilidad, alta fricción. & >25\% \\ \hline
3 & Controlado & Quick Wins activos, ledger consistente. & >10--25\% \\ \hline
5 & Fluido & Operación armónica, fricciones autogestionadas. & <10\% \\ \hline
\end{tabular}
\end{center}

Meta global: mantener IK < 15\% y evolución continua de madurez.
\\
\\
\hrule
% #################################################
% #################################################

\section*{Niveles de Implementación de Kanso}

Kanso reconoce que no todas las organizaciones poseen el mismo nivel de madurez operativa, recursos disponibles o necesidad de optimización. Por ello, el marco propone tres niveles de implementación progresivos que permiten adoptar la metodología de manera incremental, ajustándose al contexto específico de cada equipo u organización.

\section{Kanso Lite}

\textbf{Descripción}\\
Versión mínima viable de Kanso, diseñada para equipos que inician su camino hacia la reducción sistemática de fricciones o que operan con recursos limitados. Se enfoca en los fundamentos observacionales sin requerir infraestructura compleja ni roles dedicados.
\\
\\
\textbf{Perfil de Organización}
\begin{itemize}
    \item Equipos pequeños (3-10 personas)
    \item Presupuesto limitado para herramientas o roles especializados
    \item Primera aproximación a la gestión sistemática de fricciones
    \item Cultura organizacional en fase temprana de madurez
    \item Necesidad de demostrar valor antes de inversiones mayores
\end{itemize}

\textbf{Componentes Activos}
\begin{itemize}
    \item Friction Ledger simplificado: Registro básico en hoja de cálculo o documento compartido
    \item Índice Kanso básico: Cálculo mensual manual utilizando únicamente las 3-4 métricas más accesibles
    \item Reunión mensual de fricción: Sesión de 1 hora para revisar fricciones detectadas
    \item Responsable rotativo: Miembro del equipo que asume el rol de observador por ciclos de 1 mes
\end{itemize}

\textbf{Artefactos Mínimos}
\begin{center}
\small
\begin{tabular}{|p{4cm}|p{9cm}|}
\hline
\textbf{Artefacto} & \textbf{Implementación en Lite} \\ \hline
Friction Ledger & Hoja de cálculo compartida con columnas: Fecha, Descripción, Categoría, Impacto (Alto/Medio/Bajo) \\ \hline
Índice Kanso & Cálculo manual mensual con 3-4 métricas básicas (ej: tiempo promedio de resolución, tickets bloqueados, interrupciones por día) \\ \hline
Friction Analysis Matrix & No implementado (se usa evaluación cualitativa simple) \\ \hline
Harmony Board & No implementado \\ \hline
\end{tabular}
\end{center}

\textbf{Ciclo de Trabajo}\\
Ciclo mensual simplificado:
\begin{itemize}
    \item Semanas 1-3: Detección y registro continuo de fricciones
    \item Semana 4: Reunión de revisión y priorización
    \item Selección de 1-2 fricciones para resolver en el siguiente mes
\end{itemize}

\textbf{Limitaciones}
\begin{itemize}
    \item No hay análisis cuantitativo profundo (Friction RCA simplificado o ausente)
    \item Métricas limitadas y actualizadas con baja frecuencia
    \item Dependencia de percepción subjetiva para priorización
    \item Riesgo de discontinuidad por falta de rol dedicado
\end{itemize}

\textbf{Criterio de Graduación}\\
Un equipo debe considerar migrar a Kanso Core cuando:
\begin{itemize}
    \item Ha completado al menos 3 ciclos mensuales consecutivos exitosamente
    \item El Friction Ledger contiene más de 30 entradas documentadas
    \item La organización puede asignar al menos 25\% del tiempo de una persona al rol de Kanso Lead
    \item Existe demanda interna por análisis más sofisticado de fricciones
\end{itemize}

\section{Kanso Core}

\textbf{Descripción}\\
Implementación completa del marco Kanso tal como fue diseñado originalmente. Incluye todos los componentes metodológicos, roles definidos, artefactos completos y ciclos estructurados. Representa el nivel recomendado para la mayoría de las organizaciones maduras.
\\
\\
\textbf{Perfil de Organización}
\begin{itemize}
    \item Equipos medianos a grandes (10-50 personas o múltiples equipos)
    \item Capacidad de asignar un Kanso Lead dedicado (50-100\% de su tiempo)
    \item Infraestructura técnica para automatización y monitoreo
    \item Cultura organizacional abierta a la mejora continua basada en datos
    \item Presupuesto disponible para herramientas y capacitación
\end{itemize}

\textbf{Componentes Activos}
\begin{itemize}
    \item Rol formal de Kanso Lead: Persona dedicada con responsabilidad y autoridad sobre el proceso
    \item Friction Ledger completo: Sistema estructurado de registro con categorización detallada
    \item Friction Analysis Matrix: Análisis cuantitativo sistemático de Riesgo, Frecuencia y Alcance
    \item Índice Kanso: Cálculo quincenal o semanal con conjunto completo de métricas
    \item Harmony Board: Dashboard ejecutivo para visualización de tendencias
    \item Ciclo de trabajo de 2 semanas: Estructura completa según especificación estándar
\end{itemize}

\textbf{Artefactos Completos}
\begin{center}
\small
\begin{tabular}{|p{4cm}|p{9cm}|}
\hline
\textbf{Artefacto} & \textbf{Implementación en Core} \\ \hline
Friction Ledger & Base de datos o sistema de tickets estructurado con campos completos: ID, timestamp, categoría, descripción detallada, equipo afectado, tiempo perdido estimado, estado, resolución \\ \hline
Friction Analysis Matrix & Matriz cuantitativa con scoring automatizado, actualización semanal, visualización priorizada \\ \hline
Índice Kanso & Cálculo automatizado quincenal/semanal con 8-12 métricas ponderadas, tendencias históricas \\ \hline
Harmony Board & Dashboard ejecutivo en tiempo real con KPIs principales, alertas de degradación, reportes automatizados \\ \hline
Friction Reports & Reportes quincenales estructurados con análisis de tendencias y recomendaciones \\ \hline
\end{tabular}
\end{center}

\textbf{Ciclo de Trabajo}\\
Ciclo quincenal completo según especificación estándar:
\begin{itemize}
    \item Detección continua durante las 2 semanas
    \item Análisis Friction RCA paralelo
    \item Revisión en Día 6
    \item Actualización de métricas en Día 9
    \item Reporte de cierre en Día 14
\end{itemize}

\textbf{Requisitos de Infraestructura}
\begin{itemize}
    \item Sistema de tracking (JIRA, Linear, Asana o similar)
    \item Herramienta de visualización (Tableau, Grafana, Looker o equivalente)
    \item Acceso a logs y métricas de sistemas
    \item Automatización de cálculo de IK (scripts, webhooks, integraciones)
\end{itemize}

\textbf{Equipo Mínimo}
\begin{itemize}
    \item 1 Kanso Lead dedicado (50-100\% tiempo)
    \item Soporte ejecutivo (sponsor que valide decisiones estratégicas)
    \item Colaboración activa de tech leads o managers de equipo
\end{itemize}

\textbf{Beneficios Esperados}
\begin{itemize}
    \item Reducción medible de 15-30\% en fricciones recurrentes en 6 meses
    \item Mejora del IK por trimestre
    \item Visibilidad completa del estado operativo del equipo
    \item Capacidad de predicción de degradación antes de que se manifieste
    \item Cultura de mejora continua arraigada en el equipo
\end{itemize}

\section{Kanso Full}

\textbf{Descripción}\\
Extensión empresarial de Kanso Core, diseñada para organizaciones de gran escala que requieren coordinación entre múltiples equipos, geografías o unidades de negocio. Introduce componentes de gobernanza, estandarización cross-team y análisis comparativo.
\\
\\
\textbf{Perfil de Organización}
\begin{itemize}
    \item Organizaciones de 50+ personas con múltiples equipos
    \item Estructura multi-equipo con interdependencias significativas
    \item Necesidad de estandarización y benchmarking interno
    \item Presupuesto para equipo de mejora continua dedicado
    \item Madurez organizacional alta con compromiso ejecutivo sostenido
\end{itemize}

\textbf{Componentes Adicionales (sobre Core)}
\begin{itemize}
    \item Kanso Office: Equipo central de 2-4 personas dedicadas a la metodología
    \item Network de Kanso Leads: Un Kanso Lead por equipo coordinado centralmente
    \item Friction Taxonomy Corporativa: Clasificación estandarizada entre equipos
    \item Cross-Team Friction Detection: Identificación de fricciones inter-equipos
    \item Comparative Analytics: Benchmarking interno entre equipos
    \item Friction Impact Modeling: Proyecciones de ROI y análisis predictivo
    \item Integration Layer: Conexión con sistemas empresariales (ERP, HRIS, BI)
\end{itemize}

\textbf{Estructura Organizacional}

\begin{center}
\small
\begin{tabular}{|p{4cm}|p{9cm}|}
\hline
\textbf{Rol} & \textbf{Responsabilidad} \\ \hline
Chief Kanso Officer & Responsable de la estrategia de eficiencia organizacional, reporta a C-level \\ \hline
Kanso Operations Manager & Coordinación operativa del Kanso Office, estandarización de procesos \\ \hline
Kanso Data Analyst & Análisis avanzado, modelado predictivo, generación de insights \\ \hline
Kanso Leads (múltiples) & Implementación en equipos individuales, reportan matricialmente al Kanso Office \\ \hline
Kanso Champions & Evangelistas en equipos sin Kanso Lead dedicado, rol de 10-20\% tiempo \\ \hline
\end{tabular}
\end{center}

\textbf{Artefactos Extendidos}

\begin{center}
\small
\begin{tabular}{|p{4cm}|p{9cm}|}
\hline
\textbf{Artefacto} & \textbf{Implementación en Full} \\ \hline
Enterprise Friction Ledger & Agregación de fricciones de todos los equipos con taxonomía unificada, análisis de tendencias corporativas \\ \hline
Global Friction Analysis Matrix & Priorización corporativa considerando impacto cross-team y dependencias \\ \hline
Comparative IK Dashboard & Índices Kanso por equipo, división, geografía; rankings y percentiles internos \\ \hline
Friction Flow Analysis & Mapeo de cómo las fricciones se propagan entre equipos y sistemas \\ \hline
Executive Harmony Suite & Suite de dashboards para C-level con KPIs estratégicos y alertas de riesgo sistémico \\ \hline
Quarterly Efficiency Reports & Reportes trimestrales con análisis de tendencias, ROI de iniciativas, proyecciones \\ \hline
Best Practices Repository & Base de conocimiento de soluciones exitosas reutilizables \\ \hline
\end{tabular}
\end{center}

\textbf{Ciclo de Trabajo Multi-Nivel}

Kanso Full opera en tres niveles temporales sincronizados:

\begin{enumerate}
    \item Nivel Equipo (2 semanas): Cada equipo mantiene su ciclo Kanso Core independiente
    \item Nivel División (mensual): Reunión de coordinación de Kanso Leads, sincronización de prioridades cross-team
    \item Nivel Corporativo (trimestral): Revisión ejecutiva de eficiencia organizacional, decisiones estratégicas de inversión en reducción de fricción
\end{enumerate}

\textbf{Capacidades Avanzadas}

\begin{itemize}
    \item Friction Attribution Modeling: Análisis de causalidad entre fricciones y resultados de negocio (rotación, velocidad, calidad)
    \item Predictive Friction Detection: Machine learning para anticipar fricciones emergentes antes de manifestarse
    \item A/B Testing Framework: Experimentación controlada de soluciones a fricciones
    \item Friction Cost Accounting: Cálculo preciso del costo económico de cada categoría de fricción
    \item External Benchmarking: Comparación con industry standards y mejores prácticas externas
\end{itemize}

\textbf{Requisitos de Infraestructura}

\begin{itemize}
    \item Data warehouse corporativo con integración de múltiples fuentes
    \item Plataforma de BI empresarial (Tableau, Power BI, Looker)
    \item API layer para conexión con sistemas existentes
    \item Automatización avanzada (RPA para captura de métricas)
    \item Gobernanza de datos establecida (definiciones, ownership, SLAs)
\end{itemize}

\textbf{Inversión y ROI Esperado}

\begin{center}
\small
\begin{tabular}{|p{5cm}|p{8cm}|}
\hline
\textbf{Componente} & \textbf{Inversión Estimada} \\ \hline
Equipo Kanso Office & 2-4 FTEs (salarios + overhead) \\ \hline
Kanso Leads distribuidos & 0.5-1 FTE por equipo de 15-20 personas \\ \hline
Infraestructura tecnológica & \$50K-200K anuales (herramientas, integraciones) \\ \hline
Capacitación y onboarding & \$20K-50K inicial + \$10K anual \\ \hline
\textbf{Total anual} & \textbf{\$300K-800K} (org de 200-500 personas) \\ \hline
\end{tabular}
\end{center}

\textbf{ROI Proyectado:}
\begin{itemize}
    \item Reducción de 20-40\% en fricciones sistémicas en 12 meses
    \item Ahorro de 5-15\% del tiempo total de ingeniería (recuperado para trabajo productivo)
    \item Reducción de 10-25\% en rotación voluntaria por mejora de experiencia
    \item Aceleración de 15-30\% en time-to-market por eliminación de bloqueos
    \item Payback period típico: 12-18 meses
\end{itemize}

\textbf{Criterio de Adopción}\\
Una organización debe considerar Kanso Full cuando:
\begin{itemize}
    \item Tiene 3+ equipos operando exitosamente con Kanso Core
    \item Detecta fricciones recurrentes que cruzan múltiples equipos
    \item Requiere estandarización y comparabilidad entre unidades
    \item El liderazgo ejecutivo demanda visibilidad de eficiencia a nivel corporativo
    \item Existe presupuesto y commitment para sostener el Kanso Office por al menos 2 años
\end{itemize}

\hrule
% #################################################
% #################################################
\section{Matriz de Decisión: ¿Qué Nivel Implementar?}

\begin{center}
\footnotesize
\begin{tabular}{|p{3.5cm}|p{3cm}|p{3cm}|p{3cm}|}
\hline
\textbf{Criterio} & \textbf{Kanso Lite} & \textbf{Kanso Core} & \textbf{Kanso Full} \\ \hline
Tamaño de equipo & 3-10 personas & 10-50 personas & 50+ personas (múltiples equipos) \\ \hline
Recursos dedicados & 0.1 FTE rotativo & 0.5-1 FTE Kanso Lead & 2-5 FTEs Kanso Office \\ \hline
Inversión anual & <\$5K & \$50K-150K & \$300K-800K \\ \hline
Complejidad infraestructura & Mínima (spreadsheet) & Media (tracking + dashboard) & Alta (data warehouse + BI) \\ \hline
Tiempo de setup & 1-2 semanas & 1-2 meses & 3-6 meses \\ \hline
Frecuencia métricas & Mensual & Quincenal/Semanal & Continua (real-time) \\ \hline
Beneficio esperado & 5-10\% mejora & 15-30\% mejora & 30-50\% mejora \\ \hline
Payback period & Inmediato & 3-6 meses & 12-18 meses \\ \hline
\end{tabular}
\end{center}

\textbf{Recomendación General:}\\
Se sugiere comenzar siempre con Kanso Lite para validar fit cultural y compromiso organizacional, luego graduar a Core tras 3-6 meses de operación exitosa. Kanso Full debe considerarse solo cuando Core está consolidado en múltiples equipos y existe clara necesidad de coordinación cross-team.
\\
\\
\hrule
% #################################################
% #################################################
\section{Métodos de Cuantificación de Fricciones}

\subsubsection{Nivel 1 - Estimación Subjetiva (Kanso Lite)}

\begin{itemize}
    \item Auto-reporte del equipo en retrospectivas
    \item Precisión: $\pm$50\%
    \item Suficiente para identificar fricciones mayores
\end{itemize}

\subsubsection{Nivel 2 - Medición Semi-Automatizada (Kanso Core)}

\begin{itemize}
    \item Tiempo registrado en tickets tagged como ``friction-related''
    \item Análisis de logs de CI/CD (tiempo entre commits y deploy)
    \item Conteo de reuniones de coordinación
    \item Precisión: $\pm$30\%
\end{itemize}

\subsubsection{Nivel 3 - Telemetría Automatizada (Kanso Full)}

\begin{itemize}
    \item Instrumentación de sistemas con OpenTelemetry
    \item Análisis de tracing distribuido para identificar handoffs
    \item Machine learning sobre patrones de trabajo
    \item Precisión: $\pm$15\%
\end{itemize}

\subsubsection{Regla de Pragmatismo}

Comenzar siempre con Nivel 1. Subir de nivel solo cuando:

\begin{enumerate}
    \item El equipo ya tiene disciplina de registro
    \item El IK $>$ 20\% por 3 meses consecutivos
    \item Existe presupuesto para herramientas de telemetría
\end{enumerate}